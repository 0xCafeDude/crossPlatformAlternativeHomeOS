\documentclass[letterpaper,12pt]{article}
\usepackage[utf8]{inputenc}
%\usepackage[letterpaper, portrait, margin=1.0in]{geometry}
\usepackage{hyperref}

%opening
\title{Cross-platform Alternative to HomeOS Related Works (Part 2)}
\author{Christopher Becker and Shivam Kumar Sharma}

\begin{document}

\maketitle

\section{Home Automation in the Wild: Challenges and Opportunities~\cite{homeAutomation}}
\begin{description}
 \item[Title:] Home Automation in the Wild: Challenges and Opportunities
 \item[Authors:] A.J. Bernheim Brush, Bongshin Lee, Ratul Mahajan, Sharad Agarwal, Stefan Saroiu, and Colin Dixon
 \item[Venue/Source:] CHI
 \item[Year:] 2011
 \item[Summary:] Results of interviews conducted with families and individuals in regards to home automation usage.  The authors wanted positive and negative experiences from households that have used home automation for at least a year.  The authors included some households where users had installed the systems themselves and other households that had used contractors.
 \item[Similarities:] Trying to find opportunities to improve ease of use of systems and concerned with access control.
 \item[Differences:] Just getting information about the current state of home automation.
\end{description}

\section{Access Control for Home Data Sharing: Attitudes, Needs and Practices~\cite{accessControl}}
\begin{description}
 \item[Title:] Access Control for Home Data Sharing: Attitudes, Needs and Practices
 \item[Authors:] Michelle L. Mazurek, J.P. Arsenault, Joanna Bresee, Nitin Gupta, Iulia Ion, Christina Johns, Daniel Lee, Yuan Liang, Jenny Olsen, Brandon Salmon, Richard Shay, Kami Vaniea, Lujo Bauer, Lorrie Faith Cranor, Gregory R. Ganger, and Michael K. Reiter
 \item[Venue/Source:] CHI
 \item[Year:] 2010
 \item[Summary:] Results of interviews conducted with families and individuals in regards to thoughts and practices on access control for data in home environments.
 \item[Similarities:] Concerned with Access Control.
 \item[Differences:] Focus on data in general, not home automation access.
\end{description}

\section{A User Study of Policy Creation in a Flexible Access-Control System~\cite{policyCreation}}
\begin{description}
 \item[Title:] A User Study of Policy Creation in a Flexible Access-Control System
 \item[Authors:] Lujo Bauer, Lorrie Faith Cranor, Robert W. Reeder, Michael K. Reiter, and Kami Vaniea
 \item[Venue/Source:] CHI
 \item[Year:] 2008
 \item[Summary:] Examines the usage of distributed access control system named Grey.  The authors conducted a study in which multiple users used Grey in a campus environment for access control over various doors.  The authors then examined the access policies used and the difference between physical key-based access control and Grey.  The authors also looked at the policies in terms of ideal policies and the real policies that can be accomplished using physical keys and Grey.
 \item[Similarities:] Trying to determine how to have access control policies capabilities meet ideal policies as much as possible.
 \item[Differences:] A distributed system that focuses on door access.
\end{description}

\section{Catching the Z-Wave~\cite{zwavebasics}}
\begin{description}
 \item[Title:] Catching the Z-Wave
 \item[Authors:] Mikhail Galeev
 \item[Venue/Source:] EETINDIA
 \item[Year:] 2006
 \item[Summary:] The paper talks about the Z-Wave protocol. It then talks about the basic toplogy of the network. The activities that are performed by a controller. It then talks about what a developer needs to know in order to work with Z-Wave, it describes the basic command classes here.
\end{description}

\section{Security Evaluation of the Z-Wave Wireless Protocol~\cite{zwavesecurity}}
\begin{description}
 \item[Title:] Security Evaluation of the Z-Wave Wireless Protocol
 \item[Authors:] Behrang Fouladi Sahand Ghanoun
 \item[Venue/Source:] Black hat USA
 \item[Year:] 2013
 \item[Summary:] Authors conduct a security eavaluation of Z-Wave. They perform an in depth analysis of the encryption and authentication happening in the application layer of the protocol. They then perform a packet injection attack and are able to replace the shared key in a Z-wave door lock. They are then able to unlock the door without any knowledge of the original encryption keys.
\end{description}

\bibliographystyle{plain}
\bibliography{biblio}

\end{document}

