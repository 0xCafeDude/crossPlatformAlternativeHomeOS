\section{Implementation}
\label{sec:implementation}
The architecture discussed in Section~\ref{sec:arch} is implemented as a core
system, a user interface, and a set of modules. The system is written in Java.
\subsection{Core System}
\label{sec:core}
The core system is responsible for starting up the whole system. It first reads
the config file and establishes connection with the data backend. The core
system is flexible as far as the backend is concerned and can communicate with
a variety of backends including mysql, oracle, or just flat files. The data
backend provides storage and lookup capabilities for user account informaton,
group information, access control information (as described in
Section~\ref{sec:access}), and component information. Once the connection to
the data backend has been established, it loads all the protocols, devices, and
applications modules in order. The core system contains an Application Manager,
Device Manager, and a Protocol Manager to manage the various types of
components. It also keeps track of currently logged in users.
\subsection{Modules}
\label{sec:modules}
\begin{table}
\begin{center}
\begin{tabular}{| l | p{5cm} |}
\hline
start() & Used by Management Component to start the module \\ \hline
stop() & Used by Management Component to stop the module \\ \hline
setModuleId() & Used by Management Component to give the module an id \\ \hline
getModuleId() & Used by Management Component to get id of the module \\ \hline
getOfferedRoles() & Used by Management Component to get the list of roles offered by the module \\ \hline
serviceRegistered(List<Role> roles) & Used by the Management Component to tell
the module that new roles are available \\ \hline
serviceDeRegistered(List<Role> roles) & Used by the Management Componet to tell
the module that some roles are unavailable \\ \hline
\end{tabular}
\end{center}
\caption{Module Interface}
\label{tab:modulemanager}
\end{table}
\begin{table}
\begin{center}
\begin{tabular}{| l | p{5cm} |}
\hline
addModule() & adds a module to the system \\ \hline
removeModule() & removes a module from the system \\ \hline
getRole(String role) & Used by a module to request the ModuleManager to grant
access to a module offering the role sent as parameter  \\ \hline
\end{tabular}
\end{center}
\caption{ModuleManager Interface}
\label{tab:module}
\end{table}
All applications, devices and protocols implement the Module interface. The
Module interface is required for a component to interact with the Management 
Component. Table~\ref{tab:module} provides an overview of the module interface.
Table~\ref{tab:modulemanager} provides an overview of the ModuleManager.
ModuleManager is the part of the Management Component that interacts with
modules. Each module can offer some roles and some modules might need some
roles to fullfil their tasks. For example an application (which is also a
module) that can turn on a light bulb would need a role called "light bulb".
The application will ask the ModuleManager. There can be an other module called
ZipatoLightBulb that would be offering the role "light bulb". If the
application has necessary permissions the ModuleManger will allow the
application to interact with ZipatoLightBulb.
\subsection{User Interface}
\label{sec:interface}
Right now our system has a command line interface. We have designed the user
interface related parts of our system in such a way that it will be very easy
to replace the cli in the future with some other interface. The cli first asks
the user to log in. Ever user has an associated access level. 
\subsection{Access Control}
\label{sec:access}
Access control is divided into two types: core system access control and module
access control.
\subsubsection{Core System Access Control}
The access control available for the core system (discussed in
Section~\ref{sec:core}) is designed to handle administrative tasks within the
core system. Users are divided into different levels of access. Currently there
are eight levels of granularity ranging from guests with minimal access to
full administrative users which have full permissions. Administrative tasks for
the core system include user account creation and modification, group creation
and assignment, and module access control administration.

Each user is assigned an access level which is then used to determine which
commands the user can run. Each command is assigned a minimum access level.
When a user attempts to run the command, the access level of the user trying to
run the command is compared to the minimum access level of the command. If the
user's access level is greater than or equal to the minimum access level
assigned to the command, the user is able to run the command.
\subsubsection{Module Access Control}
Module access control provides access control between modules which are
discussed in more detail in Section~\ref{sec:modules}. The access control
policies consist of a set of rules. Each rule consists of: a source module, a
destination module, the days of the week the rule applies to, the start and end
times for those days, the user group the rule applies to, the priority of the
rule, and the access mode.

When a user tries to use a module (either directly or indirectly), the system
looks for a rule that applies to the use. If a rule is found where the source
and destination modules match, then the system makes sure the current time and
day of the week match the rule and the user is part of the user group assigned
to the rule. The priority portion of the rule is used in case of conflicting
access to the same module. The access mode can be either ``allow'' or ``ask.''
When the mode is ``allow,'' the interaction is automatically allowed. When the
mode is ``ask,'' a user is asked interactively to allow the interaction at the
time of access. Our system currently does not support the priority or access
mode portions of the rules.

Users are able to query the system to either figure out what modules a user can
access at a given time based on the groups he or she is in or figure out who can
access a given module. This allows the user to more reliably verify that the
access granted fits his or her expectations.
