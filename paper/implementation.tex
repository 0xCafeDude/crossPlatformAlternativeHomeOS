\section{Implementation}
\label{sec:implementation}
The architecture discussed in Section~\ref{sec:arch} is implemented as a core
system, a user interface, and a set of modules.
\subsection{Core System}
\label{sec:core}
\subsection{Modules}
\label{sec:modules}
\subsection{User Interface}
\label{sec:interface}
\subsection{Access Control}
\label{sec:access}
Access control is divided into two types: core system access control and module
access control.
\subsubsection{Core System Access Control}
The access control available for the core system (discussed in
Section~\ref{sec:core}) is designed to handle administrative tasks within the
core system. Users are divided into different levels of access. Currently there
are eight levels of granularity ranging from guests with minimal access to
full administrative users which have full permissions. Administrative tasks for
the core system include user account creation and modification, group creation
and assignment, and module access control administration.

Each user is assigned an access level which is then used to determine which
commands the user can run. Each command is assigned a minimum access level.
When a user attempts to run the command, the access level of the user trying to
run the command is compared to the minimum access level of the command. If the
user's access level is greater than or equal to the minimum access level
assigned to the command, the user is able to run the command.
\subsubsection{Module Access Control}
Module access control provides access control between modules which are
discussed in more detail in Section~\ref{sec:modules}. The access control
policies consist of a set of rules. Each rule consists of: a source module, a
destination module, the days of the week the rule applies to, the start and end
times for those days, the user group the rule applies to, the priority of the
rule, and the access mode.

When a user tries to use a module (either directly or indirectly), the system
looks for a rule that applies to the use. If a rule is found where the source
and destination modules match, then the system makes sure the current time and
day of the week match the rule and the user is part of the user group assigned
to the rule. The priority portion of the rule is used in case of conflicting
access to the same module. The access mode can be either ``allow'' or ``ask.''
When the mode is ``allow,'' the interaction is automatically allowed. When the
mode is ``ask,'' a user is asked interactively to allow the interaction at the
time of access. Our system currently does not support the priority or access
mode portions of the rules.

Users are able to query the system to either figure out what modules a user can
access at a given time based on the groups he or she is in or figure out who can
access a given module. This allows the user to more reliably verify that the
access granted fits his or her expectations.
