\section{Introduction}
\label{sec:intro}
Networked ``smart home'' devices such as remotely controlled lights, cameras,
thermostats, and locks have become relatively inexpensive and widely available.
While, in theory, this should allow the proliferation of these devices as well
as the ability to use them in a wide range of scenarios such as adjusting the
climate control settings of a room based on occupancy levels or remotely
controlling a device such as a camera through a smart phone, in practice this is
not the case. The overhead of managing and extending the current systems is
generally not reasonable for any but expert hobbyists and the rich. As part of
the management of a system, access control is often difficult to correctly
configure, especially in systems with many devices and many users. This leads to
security and privacy issues which can result from misconfigured access control
rules.

Some recent work has been done to try to rectify this disparity by providing a
PC-like abstraction to easily manage and extend systems to include devices and
allow a wide range of scenarios. One specific system, HomeOS~\cite{homeOS}, was
developed by Microsoft Research. HomeOS is designed to be a system which
presents devices as peripherals, uses applications to allow cross-device tasks,
provides a user interface designed to be used in a home environment, and an easy
to understand access control system. However, HomeOS only runs on Windows.

We present an alternative to HomeOS which provides cross-platform support. Our
main contribution is providing a cross-platform system designed to have similar
capabilities as HomeOS which uses open source components. We also have made
minor changes to the system which we feel will improve the overall system.

The remainder of the paper is structured as follows: In
Section~\ref{sec:related} we introduce related work, focusing on HomeOS. In
Section~\ref{sec:arch} we provide an overview of the system architecture. In
Section~\ref{sec:implementation} we provide details about our implementation and
current status. We conclude with Section~\ref{sec:future} discussing future
work.