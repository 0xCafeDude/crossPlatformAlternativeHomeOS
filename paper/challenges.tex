\section{Challenges}
\label{sec:challenges}
While developing the system we had to overcome a number of challenges. We have 
mentioned them below.

The first challange we faced was in installation of the drivers for the zwave
controller. The drivers were not compatible with our kernel version ( Ubuntu 
3.13.0-43-generic). We started modifying the existing driver source but 
eventually we were able to find drivers that had been modified to work with our
system.

The second challenge we faced was in building the Java wrappers for the
OpenZWave library. The build process for the Java wrappers requires the path to
the directory containing the source code for OpenZWave. If the OpenZWave
directory contains the source code as well as the binaries, the building of the
Java wrappers failed. After a lot of hit and trail we were luckily able to find
this out and fix it.

The third challenge we faced was in changing the color of the light bulb. The
OpenZWave library doesn't support that yet. We decided to write our own code to
do that. We were not able to find out the exact message that needs to be sent to
the bulb in order to change its color. The original Z-Wave library is
proprietary and there is no freely available documentation for it. Most of the 
OpenZWave library has been developed by reverse engineering and hit and trial.
The OpenZWave forum helped us here and told us that device needs to be sent an
RGB value to change its color. The exact byte sequence that needs to be sent
will contain the following
0x20x"Red color value"0x3"Green color value"0x4"Blue color value".

The last challenge was because of Java not allowing us to free memory on demand.
When ManagementComponent approves the request of Module1 to communicate with
Module2 it sends an instance of Module2 to the requesting instance of Module1.
In the original HomeOS paper if the ManagementComponent later decides to revoke
Module1's capability to access Module2 it simply frees the memory occupied by
the instance of Module2 this is something that we cannot do in Java. Hence we 
had to redesign our system and force all communication that can happen between
modules to go through the ManagementComponent.  
