\section{Challenges}
\label{sec:challenges}
While developing the system we had to overcome a number of challenges. We have 
mentioned some of them below.

The first challenge we faced was in installation of the drivers for the Z-Wave
controller. The drivers were not compatible with our kernel version (Ubuntu 
3.13.0-43-generic). We started modifying the existing driver source, but 
eventually we were able to find drivers that had been modified to work with our
system.

The second challenge we faced was in building the Java wrappers for the
OpenZWave library. The build process for the Java wrappers requires the path to
the directory containing the source code for OpenZWave. If the OpenZWave
source code library had been previously built using the included Makefile, the
building of the Java wrappers failed. After a lot of trial and error, we were
able to find this out and fix it.

The third challenge we faced was in changing the color of the light bulb. The
OpenZWave library doesn't support that yet. We decided to write our own code to
do that. We were not able to find out the exact message that needs to be sent to
the bulb in order to change its color. The original Z-Wave library is
proprietary and there is no freely available documentation for it. Most of the 
OpenZWave library has been developed by reverse engineering and trial and error.
The OpenZWave forum helped us here and told us that the device needs to be sent
an RGB value to change its color. The exact byte sequence that needs to be sent
will contain the following 0x02 ``Red color value'' 0x03 ``Green color value''
0x04 ``Blue color value,'' where the red, green, and blue color values are
between 0 and 255 inclusive. For example, to change the light color to red, the
following byte sequence would be sent: 0x02 0xFF 0x03 0x00 0x04 0x00.

The last challenge was dealing with the fact that Java does not allow us to free
memory on demand. When the Management Component approves the request of a Module
(Module1) to communicate with another module (Module2), it sends a reference for
Module2 to Module1. In the original HomeOS paper, if the Management Component
later decides to revoke Module1's capability to access Module2, it simply frees
the memory occupied by the instance of Module2. This is something that we
cannot do in Java. Hence, we had to redesign our system and force all
communication that can happen between modules to go through the
Management Component.
