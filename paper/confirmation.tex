\documentclass[letterpaper,12pt]{article}
\usepackage[utf8]{inputenc}
\usepackage[letterpaper, portrait, margin=1.0in]{geometry}
\usepackage{hyperref}

%opening
\title{Cross-platform Alternative to HomeOS}
\author{Christopher Becker and Shivam Kumar Sharma}

\begin{document}

\maketitle

\begin{abstract}
Networked ``smart home'' devices such as remotely controlled lights, cameras,
thermostats, and locks have become relatively inexpensive and widely available.
While, in theory, this should allow the proliferation of these devices as well
as the ability to use them in a wide range of scenarios such as adjusting the
climate control settings of a room based on occupancy levels or remotely
controlling a device such as a camera through a smart phone, in practice this is
not the case. The overhead of managing and extending the current systems is
generally not reasonable for any but expert hobbyists and the rich. As part of
the management of a system, access control is often difficult to correctly
configure, especially in systems with many devices and many users. This leads to
security and privacy issues which can result from misconfigured access control
rules.

Some recent work has been done to try to rectify this disparity by providing a
PC-like abstraction to easily manage and extend systems to include devices and
allow a wide range of scenarios. One specific system, HomeOS, was developed by
Microsoft Research. HomeOS is designed to be a system which presents devices as
peripherals, uses applications to allow cross-device tasks, provides a user
interface designed to be used in a home environment, and an easy to understand
access control system. However, HomeOS only runs on Windows.  We present an
alternative to HomeOS which provides cross-platform support.
\end{abstract}


\section{Presentation Time Confirmation}
We are in the 2:30-4:00 presentation slot on Tuesday, December 9th.  We are
listed third so our specific presentation time should be around 3:30.  We are
scheduled to present in the Merrill Engineering Building (MEB) Large Conference
Room (LCR).

%\bibliographystyle{plain}
%\bibliography{biblio}

\end{document}

 
