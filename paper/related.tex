\section{Related Work}
\label{sec:related}
There is much research in the direction of making home automation easier by
Microsoft. HomeOS was developed by Microsoft Research. It is designed to provide
a PC-like abstraction for home automation devices. The system presents devices
as peripherals and uses applications to allow cross-device tasks. The user
interface for the system is designed for a home environment and is designed to
provide an easy to understand access control system. The system itself is
written in C\# and only runs on the Windows operating system. Since the original
system, Microsoft Research has continued to use it for other research. Examples
of this continued research are HomeLab~\cite{homeLab} and Lab of
Things~\cite{labOfThings}.

HomeLab is a shared infrastructure for conducting field studies for home
technology. HomeLab consists of a number of homes all containing different home
devices connected to PCs running HomeOS. A number of research groups can use
this infrastructure for conducting experiments on their device of interests.
HomeLab will allow easy sharing and co-ordination of research effort between the
different groups and will also allow them to focus on the actual experiment
rather than on the low level details of device connectivity etc. HomeLab also
allows groups to remotely update, monitor, and collect data from HomeOS.

Lab of Things consists of HomeOS and a number of cloud services. One of these
services is the Home Store from which a user can download drivers for different
devices.

Other research has occurred with relation to the security, privacy, usage, and
access control for home automation and home environments. Denning, Kohno, and
Levy~\cite{modernHome} provides an overview of security and privacy issues
related to home automation and Internet of Things devices. Fouladi and
Ghanoun~\cite{zwavesecurity} provides a security evaluation of the Z-Wave
communication used by some home automation devices. They perform an in depth
analysis of the encryption and authentication happening in the application layer
of the protocol.

Brush et al.~\cite{homeAutomation} discusses the results of interviews conducted
with families and individuals in regards to home automation usage. The
interviews were designed to collect both positive and negative experiences from
households that have used home automation for at least a year. The households
interviewed varied in ability and whether or not they had installed the systems
themselves or had used contractors.

Bauer, Cranor, Reeder, Reiter, and Vaniea~\cite{policyCreation} examines the
usage of a distributed access control system in a campus environment to study
the differences between physical key-based access control and an access control
system as well as the differences between ideal access control policies and real
access control policies used by the different types of access control systems.
Mazurek et al.~\cite{accessControl} discusses the results of interviews
conducted with families and individuals in regards to thoughts and practices for
access control for data in home environments. Ur, Jung, and
Schechter~\cite{currentState} provides an analysis of access control needs for
an LED color-changing light bulb, wireless scale, and wireless keypad door lock
with suggestions for access control requirements for home automation systems.
