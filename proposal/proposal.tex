\documentclass[letterpaper,12pt]{article}
\usepackage[utf8]{inputenc}
\usepackage[letterpaper, portrait, margin=1.0in]{geometry}
\usepackage{hyperref}

%opening
\title{Cross-platform Alternative to HomeOS}
\author{Christopher Becker and Shivam Kumar Sharma}

\begin{document}

\maketitle

\section{Problem Definition and Motivation}
Our main goal is to create a cross-platform alternative of HomeOS~\cite{homeOS}. HomeOS is a PC-like abstraction for all the networked devices at home. It presents network devices as peripherals with abstract interfaces, enables cross-device tasks via applications written against these interfaces, and provides users a management interface designed for the home environment. The management interface is designed to provide improved access control. HomeOS runs only on Windows. We hope to create an alternative that is platform-independent and, in the process of doing so, modify aspects of it, particularly the access control module, or create something new all together.

\section{Why we chose this topic}
Both of us are interested in working on the Internet of Things. We also want to do something that is programming-oriented and has some tangible outcomes. Creating a cross-platform alternative to HomeOS fulfills our above mentioned criteria. One of the major contributions of HomeOS is its access control module. Since we have a mutual interest in security, it is an added bonus for us.

\section{Division of Work}
We will be working closely on most of the tasks. As of now, we don't have a clear picture of all that needs to be done. Once that is established, we will divide the work accordingly.

\section{Approach}
Our short term goals are to read more about HomeOS, understand the existing code, and create a system architecture of the alternative that we are going to develop. Once we have created the system architecture, we will start writing the code. Initially, we plan to write modules for two different devices. One of the devices will be one that requires minimal access control (e.g., lights) and the other device will be something that has high requirements for access control (e.g., door locks).

\section{Evaluation}
Our evaluation will mainly involve experimentation and testing. We will check the code to determine whether or not it works. If it works, we will further check to determine whether or not it runs on all the major platforms.

\section{Obstacles}
The biggest hurdle that we will face will be interfacing with various protocol stacks and device drivers. We will attempt to interface with one simple device if that happens. We are aware that we might face issues in running the code on different platforms. If that happens, we will focus on writing code only for Ubuntu 14.04.

\section{Related Work}
There is a lot of research in this direction by Microsoft.  HomeOS was developed by Microsoft Research.  Since the original system, Microsoft Research has continued to use it for other research. Examples of this continued research are HomeLab~\cite{homeLab} and Lab of Things~\cite{labOfThings}. Lab of Things consists of HomeOS and a number of cloud services. One of these services is the Home Store from which a user can download drivers for different devices. As of now, we are creating a replica/alternative of the HomeOS. Our hope is that as we continue developing our replica, our approach will start deviating from that of HomeOS. HomeLab is a shared infrastructure for conducting field studies for home technology. HomeLab consists of a number of homes all containing different home devices connected to PC's running HomeOS. A number of research groups can use this infrastructure for conducting experiments on their device of interests. HomeLab will allow easy sharing and co-ordination of research effort between the different groups and will also allow them to focus on the actual experiment rather than on the low level details of fevice connectivity etc. HomeLab also allows groups to remotely update, monitor and collect data from HomeOS.

\section{Milestones}
These are the list of goals that we have thought of as of now.
\begin{enumerate}
 \item Create a system architecture.
 \item Develop device drivers for the devices that we have access to.
 \item Develop the access control module.
 \item Develop the user interface.
 \item Test.
 \item If testing is successful, we will try to add new devices and/or try to develop cross device features.
\end{enumerate}

\bibliographystyle{plain}
\bibliography{biblio}

\end{document}

